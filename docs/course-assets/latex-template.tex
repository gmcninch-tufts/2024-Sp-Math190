\documentclass{article}
   
\usepackage{graphicx, color}
\usepackage{palatino, mathpazo}
\usepackage{enumerate, mathtools}
\usepackage{hyperref}
\usepackage[margin=2cm]{geometry}
\usepackage{minted}
\usemintedstyle{tango}
\usepackage[svgnames]{xcolor}
%
% Edit me!!
%
\newcommand{\myname}{MyName}
\newcommand{\assignment}{Homework \(n\)}
%
%
\newcounter{problem}
\newenvironment{problem}{
\refstepcounter{problem}
\noindent
{\color{NavyBlue}\textbf{Problem \theproblem.}}

\noindent
\hspace{.02\textwidth}
\begin{minipage}[t]{.98\textwidth}}
{\end{minipage}
\vspace{5mm}}


%% ----------------------------------------------------
%% topline is a bit like a function of 3 "variables", 
%% referred to as #1, #2, and #3
%% ----------------------------------------------------
\newcommand{\topline}[3]{
\noindent
{
\color{NavyBlue}
\begin{minipage}[t]{.35\textwidth}
#1
\end{minipage}
\begin{minipage}[t]{.35\textwidth}
#2
\end{minipage}
\begin{minipage}[t]{.30\textwidth}
#3
\end{minipage}
\hrule
}
}

%% ------------------------------------------------------
%% end of topmatter
%%-------------------------------------------------------

\begin{document}

\topline{Math 087 - Fall 2020}{Assignment 0}{\myname}

\vspace{2cm}

\begin{problem}
Here is the first homework problem solution.
\[\int_{-\infty}^\infty e^{-x^2}dx = \sqrt{\pi}.\]
\end{problem}

\begin{problem}
Here is the second homework problem solution.
It has some parts:
\begin{enumerate}[(a)]
\item One aspect of this problem.
  
  Some further discussion.

  Still more...
\item Another aspect
\end{enumerate}
And it has a 
\href{http://www.tufts.edu}{URL reference (Tufts)}.
%% ----------------------------------------------------     
%% the command \href and the behavior of
%% \begin{enumerate}[(a)] ...
%% depend on some of the packages we loaded in the
%% topmatter.
%% ----------------------------------------------------   
\end{problem}

\begin{problem}
  Here we include some computer code.

  \begin{minted}[bgcolor=Lavender]{Python}
    def square(x): 
    return x*x
    
    map(square,[1,2,3])
  \end{minted}
\end{problem}
\end{document}
